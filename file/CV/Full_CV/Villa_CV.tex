%%%%%%%%%%%%%%%%%%%%%%%%%%%%%%%%%%%%%%%%%
% Medium Length Professional CV
% LaTeX Template
% Version 2.0 (8/5/13)
%
% This template has been downloaded from:
% http://www.LaTeXTemplates.com
%
% Original author:
% Trey Hunner (http://www.treyhunner.com/)
%
% Important note:
% This template requires the resume.cls file to be in the same directory as the
% .tex file. The resume.cls file provides the resume style used for structuring the
% document.
%
%%%%%%%%%%%%%%%%%%%%%%%%%%%%%%%%%%%%%%%%%

%----------------------------------------------------------------------------------------
%	PACKAGES AND OTHER DOCUMENT CONFIGURATIONS
%----------------------------------------------------------------------------------------

\documentclass{resume} % Use the custom resume.cls style

\usepackage[left=0.75in,top=0.6in,right=0.75in,bottom=0.6in]{geometry} % Document margins
\newcommand{\tab}[1]{\hspace{.2667\textwidth}\rlap{#1}}
\newcommand{\itab}[1]{\hspace{0em}\rlap{#1}}
\name{Chiara Villa} 
\address{PhD student \\ University of St Andrews \\ Mathematical Biology} 

\begin{document}

%----------------------------------------------------------------------------------------
%	EDUCATION SECTION
%----------------------------------------------------------------------------------------
\begin{tabular}{ @{} >{\bfseries}l @{\hspace{6ex}} l }
Email &  cv23@st-andrews.ac.uk \\
Phone & +44 7534774146 \\
Website & http://www.mcs.st-and.ac.uk/~cv23/ \\
%Languages &  Italian (native), English (C2), French (B2), Brasilian Portuguese (A2) \\
%Programming &  {\em Proficient:} MATLAB, Python, LaTeX, Fortran90, {\em Basic:} Maple, R, HTML5 \\
\end{tabular}

%{\em First year PhD student working in the StAMBio group at the University of St Andrews. Reliable, well- organised, determined and hard working. Particularly interested in mathematical modelling of biological systems, focusing on systems of PDEs to describe tumour development and numerical simulations. }

\begin{rSection}{Education}

{\bf Phd, Mathematics}, {\em University of St Andrews} \hfill {Sep 2018 - Present} \\
{Supervisors: Prof Mark Chaplain, Dr Tommaso Lorenzi \\
Research topic: Mathematical Modelling of Tumour Growth and Anti-cancer Therapy \\ Funding awarded by the School of Mathematics and Statistics, University of St Andrews} \\
{\bf Scottish Mathematical Sciences Training Center} \hfill {Sep 2018 - Sep 2019} \\
{Graduate courses in Continuum Mechanics, Numerical Methods, Mathematical Biology and Physiology} \\
{\bf MMaths, Applied Mathematics}, {\em University of St Andrews} \hfill {2014 - 2018}\\
{First Class Honours awarded \\
Academic Prizes: The Principal’s Scholarship for Academic Excellence, Dean’s list \\
Final project: Mathematical Modelling of Tumour-Induced Angiogenesis} 
\\
{\bf Scientific High School}, {\em Istituto Internazionale Edoardo Agnelli}, Torino (IT) \hfill {2009 - 2014}\\
{Diploma awarded: grade 100/100 \\
Maths and Physics extra-curricular activities: Physics Olympic games, “Festa della matematica”, Maths Olympic games and Archimedes’ games, Maths summer internship in Bardonecchia} 
\\
{\bf Study abroad}, {\em Epsom Girls Grammar School}, Auckland (NZ) \hfill {Summer 2011}\\
{Award for “Highly commended in Mathematics” during mid-term exams} 

\end{rSection}
%-----------------------------%


\begin{rSection}{Work Experience}
{\bf Bar Staff} \hfill {2017 - 2018} \\
{{\em University of St Andrews Students’ Association}, St Andrews (UK)} \\
{\bf Visiting Days Student Ambassador} \hfill {Spring 2018} \\
{{\em School of Mathematics and Statistics, University of St Andrews}, St Andrews (UK)} \\
{\bf Undergraduate Summer Research Internship} \hfill {Summer 2017} \\
{{\em StAMBio group, School of Mathematics and Statistics, University of St Andrews}, St Andrews (UK) \\ Topic: Mathematical modelling of spatio-temporal evolutionary dynamics of cancer cells focusing on the phenotypic landscape of a solid tumor (Numerical simulations in Matlab)} \\
{\bf Complex Systems Biology Research} \hfill {Summer 2016} \\
{{\em Prof Michele Caselle, Dipartimento di Fisica, Universit{\`a} degli Studi di Torino}, Torino (IT)\\ Topic: Role of ohnolog genes in regulatory networks, with a focus on co-regulation and self-regulation of paralogue pairs (Data analysis in Python)} \\
{\bf Au-pair Childcare} \hfill {Summer 2015} \\
{{\em Full-time nanny in a family with 3 kids}, Genova (IT)} \\
{\bf Group Leader in Local Community} \hfill {2011 - 2014} \\
{{\em Volunteer group leader with GiOC}, Torino (IT) \\ Preparation and coordination of group discussions with educational and aggregative purpose, organisation and supervision of all activities in summer camps} \\
{\bf ONLUS Volunteer} \hfill {Summer 2013} \\
{{\em Volunteer for the ONLUS foundation ``Aiutare i Bambini"}, Ara\c{\'c}uai (BR) \\ Supervision of the project to support the Assant Centre and the 45 children assigned to it from the juvenile Court, preparation of documents for long-distance adoption} \\
{\bf Private Tutor} \hfill {2010 - 2013} \\
{{\em Private tutor in Maths and English to secondary school pupils}, Torino (IT)} \\
{\bf English Teacher} \hfill {Summer 2012} \\
{{\em Primary school English teacher in a summer camp}, Su Zhou (CHN)\\ {English language lessons in three classes with pupils aged 8-9, 9-10, 10-11 respectively, afternoon after-school management and recreational activities for primary and secondary school kids}} 
\end{rSection}  
%-----------------------------%

\begin{rSection}{University Teaching and Marking}
All available student feedback data is included and reported on a scale of 1 (excellent) to 5 (poor) in the categories
of Explanation (E), Organisation (O) and Availability (A). 

{\bf MT2000 Computing Workshop}, {\em University of St Andrews} \hfill {Autumn 2020} \\
{Demonstrator in Python Computing Workshop for 2000 level courses} \\
{\bf MT2501 Linear Mathematics}, {\em University of St Andrews} \hfill {Autumn 2019} \\
{Tutor of two groups (11 students each)} -- E=1.44, O=1.33, A=1.33 \\
{\bf MT2000 Computing Workshop}, {\em University of St Andrews} \hfill {Autumn 2019} \\
{Demonstrator in Python Computing Workshop for 2000 level courses} \\
{\bf MT2507 Mathematical Modelling}, {\em University of St Andrews} \hfill {Spring 2018} \\
{Tutor of two groups (11 students each) -- E=1.45, O=1.85, A=1.45} \\
{Demonstrator of three groups (50 students each)} \\
{\bf MT2503 Multivariate Calculus}, {\em University of St Andrews} \hfill {Autumn 2018} \\
{Tutor of two groups (10-12 students each) -- E=1.17, O=1.5, A=1.17} \\
{\bf MT2504 Combinatorics and Probability}, {\em University of St Andrews} \hfill {Autumn 2018} \\
{Computer projects marker} 
\end{rSection}
%-----------------------------%

\begin{rSection}{Talks}
{\bf SoftMech Workshop}, {\em Online} \hfill {June 2021} \\
{``Mechanical models of pattern and form in biological tissues: the role of stress-strain constitutive equations"}  \\
{\bf Mathematical Biology on the Mediterranean Coast}, {\em Online} \hfill {May 2021} \\
{``Mathematical modelling of early stages vasculogenesis and cell-matrix interactions."}  \\
{\bf StAMBio Seminar}, {\em Online} \hfill {April 2021} \\
{``A mathematical model of endothelial progenitor cell cluster formation during the early stages of vasculogenesis''} \\
{\bf Mathematical Population Dynamics, Ecology and Evolution}, {\em Online} \hfill {April 2021} \\
{``Modelling the adaptive dynamics of space- and phenotype-structured populations of cancer cells"}  \\
{\bf StAMBio Seminar}, {\em Online} \hfill {April 2021} \\
{``A mathematical model of endothelial progenitor cell cluster formation during the early stages of vasculogenesis"}  \\
{\bf SoftMech Seminar}, {\em Online} \hfill {March 2021} \\
{``Mechanical models of pattern and form in biological tissues: the role of stress-strain constitutive equations"}  \\
{\bf StAMBio Seminar}, {\em Online} \hfill {July 2020} \\
{``Mathematical modelling of early-stages cluster-based vasculogenesis"}  \\
{\bf Interplay between Oncology, Mathematics and Numerics }, {\em Online} \hfill {June 2020} \\
{``Modelling the emergence of pre-treatment phenotypic heterogeneity in vascularised tumours"}  \\
{\bf Postgraduate Interdisciplinary Mathematics Symposium}, {\em The Burn House} \hfill {January 2020} \\
{``Pattern formation in linear viscoelastic materials"}  \\
{\bf School of Mathematics and Statistics Research Day}, {\em University of St Andrews} \hfill {January 2020} \\
{``Pattern formation in linear viscoelastic materials"}  \\
{\bf Scottish Mathematical Biology Forum}, {\em ICMS} \hfill {December 2019} \\
{``Modelling the emergence of phenotypic heterogeneity in vascularised tumours"}  \\
{\bf Visit to Laboratoire Jacques-Louis Lions}, {\em Sorbonne University} \hfill {November 2019} \\
{``Modelling the emergence of phenotypic heterogeneity in vascularised tumours"}  \\
{\bf EMS Postgraduate Meeting}, {\em The Burn House} \hfill {May 2019} \\
{``Models of viscoelasticity and their pattern formation potential"}  \\
{\bf StAMBio Internal Seminar}, {\em University of St Andrews} \hfill {April 2019} \\
{``Assessing the impact of tissue vascularisation on intratumour heterogeneity using a formal Hamilton-Jacobi approach"} \\
{\bf Postgraduate Interdisciplinary Mathematics Symposium}, {\em The Burn House} \hfill {January 2019} \\
{``A snapshot of Mathematical Biology"} \\
{\bf MT234 Research and Party}, Outreach event, {\em University of St Andrews} \hfill {November 2018} \\
{``Cancer modelling: towards virtual biopsies"} \\
{\bf Master thesis defence},  {\em University of St Andrews} \hfill {April 2018} \\
{``Mathematical modelling of tumour-induced angiogenesis"} \\
{\bf Undergraduate Summer Research Conference}, {\em University of St Andrews} \hfill {November 2017} \\
{``Dissecting cancer through mathematics: how the tumour microvasculature and microenvironment influence the eco-evolutionary dynamics of cancer cells"} 
\\
{\bf Reading Party}, {\em University of St Andrews} \hfill {February 2017} \\
{``Mathematical modelling of avascular tumour growth"} 
\end{rSection}
%-----------------------------%

\begin{rSection}{Conferences, Workshops and forums}
{\bf SoftMech Workshop} \hfill {June 2021} \\
{{\em University of St Andrews}, Online conference} \\
{\bf Mathematical Biology on the Mediterranean Coast} \hfill {May 2021} \\
{{\em Sorbonne University (LJLL)}, Online conference} \\
{\bf Mathematical Population Dynamics, Ecology and Evolution} \hfill {April 2021} \\
{{\em CIRM}, Online conference} \\
{\bf British Applied Mathematics Colloquium} \hfill {April 2021} \\
{{\em University of Glasgow}, Online conference} \\
{\bf Postgraduate Interdisciplinary Mathematics Symposium} \hfill {January 2021} \\
{{\em  University of St Andrews}, Online conference} \\
{\bf Society for Mathematical Biology (Awarded SMBdevBio Poster Prize 1)} \hfill {August 2020} \\
{Online conference} \\
{\bf Society for Mathematical Biology \& European Society for Mathematical and Theoretical Biology (Cancelled due to COVID-19)} \hfill {August 2020} \\
{%{\em Heidelberg University}, Cancelled due to COVID-19 \\
Mini Symposium invited speaker, LMS ECR Travel Grant awarded (\pounds 500)} \\
{\bf Interplay between Oncology, Mathematics and Numerics (Invited speaker)} \hfill {June 2020} \\
{{\em Sorbonne University (LJLL), Inserm, University of Poitiers}, Online conference} \\
{\bf Postgraduate Interdisciplinary Mathematics Symposium (Organiser)} \hfill {January 2020} \\
{{\em  School of Mathematics and Statistics}, The Burn House, Edzell} \\
{\bf School of Mathematics and Statistics Research Day} \hfill {January 2020} \\
{{\em  School of Mathematics and Statistics}, St Andrews} \\
{\bf Scottish Mathematical Biology Forum (Invited speaker)} \hfill {December 2019} \\
{{\em  Maxwell Institute for Mathematical Sciences}, ICMS, Edinburgh} \\
{\bf Modeling, analysis and simulation} \hfill {November 2019} \\
{{\em  Laboratoire Jacques-Louis Lions}, Sorbonne University, Paris} \\
{\bf EMS Postgraduate Meeting} \hfill {May 2019} \\
{{\em  Edinburgh Mathematical Society}, The Burn House, Edzell} \\
{\bf Computational Approaches in Mathematical Biology} \hfill {May 2019} \\
{{\em  University of Dundee}, Dundee} \\
{\bf Research School: PDEs in Mathematical Biology: Modelling and Analysis} \hfill {May 2019} \\
{{\em  London Mathematical Society \& Clay Mathematics Institute}, ICMS, Edinburgh} \\
{\bf British Applied Mathematics Colloquium} \hfill {April 2019} \\
{{\em University of Bath}, Bath} \\
{\bf Postgraduate Interdisciplinary Mathematics Symposium} \hfill {January 2019} \\
{{\em University of St Andrews}, The Burn House, Edzell} \\
{\bf School of Mathematics and Statistics Research Day} \hfill {January 2019} \\
{{\em University of St Andrews}, St Andrews} \\
{\bf Scottish Mathematical Biology Forum} \hfill {December 2018} \\
{{\em Maxwell Institute for Mathematical Sciences}, Edinburgh} \\
{\bf Scottish Mathematical Training Center Symposium} \hfill {October 2018} \\
{{\em SMSTC}, Perth} \\
{\bf Undergraduate Summer Research Conference} \hfill {November 2017} \\
{{\em School of Mathematics and Statistics}, St Andrews} 
\end{rSection}
%-----------------------------%
%\clearpage

\begin{rSection}{Skills and Activities}
{\bf Languages} \\
{Italian -- native speaker \\
English -- advanced (C2): CAE–C1 (03/13) and TOEFL (12/13) certificates \\
French – intermediate (B1): DELF B1 (08/12) certificate and attendance of several intensive courses in Torino (Alliance Française), Nice (International House) and St Andrews (evening language courses) \\ Brazilian Portuguese – beginner (A1): attendance of a course with mother tongue teacher} \\
{\bf Programming and Computer Skills} \\
{{\em Proficient:} MATLAB, Python, LaTeX, Fortran90 \\ {\em Basic:} COMSOL, Maple, R, HTML5 \\  Microsoft Office tools (ECDL Full Certificate), video editing (Movie Maker, AVS Video Editor)} \\
{\bf Personal interests} \\
{Acoustic guitar, photography, travelling, swimming, hip-hop dancing, figure ice-skating, yoga} \\
{\bf Outreach activity} \\
{Piscopia Society outreach: % (11/20)
 testimonial to encourage female/non-binary students considering a PhD } 
\end{rSection}
%-----------------------------%

\begin{rSection}{Professional responsibilities}
{\bf School of Mathematics and Statistics}, {\em StAMBio online seminars organiser} \hfill {Sep 2020 - today} 
{\bf School of Mathematics and Statistics}, {\em Mentor in Peer Mentoring scheme} \hfill {Sep 2018 - today} 
{\bf School of Mathematics and Statistics}, {\em PGR Rep \& PGR Exec Rep} \hfill {Sep 2018 - Sep 2019} \\
{\bf Scottish Mathematical Sciences Training Center}, {\em UoSA Student Rep} \hfill {Sep 2018 - Sep 2019} 
\end{rSection}
%-----------------------------%

\begin{rSection}{Publications}
{\bf 5.} {\bf C. Villa}, A. Gerisch, M.A.J. Chaplain, A mathematical model of endothelial progenitor cell cluster formation during the early stages of vasculogenesis, preprint arXiv:2105.11221, 2021 \\
{\bf 4.} {\bf C. Villa}, M.A.J. Chaplain, A. Gerisch, T. Lorenzi, {Mechanical models of pattern and form in biological tissues: the role of stress-strain constitutive equations}, Bulletin of Mathematical Biology, in press, 2021 \\
{\bf 3.} {\bf C. Villa}, M.A.J. Chaplain, T. Lorenzi, {Evolutionary dynamics in vascularised tumours under chemotherapy: Mathematical modelling, asymptotic analysis and numerical simulations}, Vietnam Journal of Mathematics, 49, 143–167, 2021 \\
{\bf 2.} {\bf C. Villa}, M.A.J. Chaplain, T. Lorenzi, {Modelling phenotypic heterogeneity in vascularised tumours}, SIAM Journal on Applied Mathematics, 81, 434–453, 2021 \\
{\bf 1.} T. Lorenzi, F.R. Macfarlane, {\bf C. Villa}, {Discrete and continuum models for the evolutionary and spatial dynamics of cancer: a very short introduction through two case studies}, (pp. 359-380) in Trends in Biomathematics: Modeling Cells, Flows, Epidemics, and the Environment, Ed. R. Mondaini, Springer, Cham, 2019
\end{rSection}
\end{document}
