%%%%%%%%%%%%%%%%%%%%%%%%%%%%%%%%%%%%%%%%%
% Medium Length Professional CV
% LaTeX Template
% Version 2.0 (8/5/13)
%
% This template has been downloaded from:
% http://www.LaTeXTemplates.com
%
% Original author:
% Trey Hunner (http://www.treyhunner.com/)
%
% Important note:
% This template requires the resume.cls file to be in the same directory as the
% .tex file. The resume.cls file provides the resume style used for structuring the
% document.
%
%%%%%%%%%%%%%%%%%%%%%%%%%%%%%%%%%%%%%%%%%

%----------------------------------------------------------------------------------------
%	PACKAGES AND OTHER DOCUMENT CONFIGURATIONS
%----------------------------------------------------------------------------------------

\documentclass{resume} % Use the custom resume.cls style

\usepackage[left=0.75in,top=0.6in,right=0.75in,bottom=0.6in]{geometry} % Document margins
\newcommand{\tab}[1]{\hspace{.2667\textwidth}\rlap{#1}}
\newcommand{\itab}[1]{\hspace{0em}\rlap{#1}}
\name{Chiara Villa} 
\usepackage{tabularx}
\usepackage{textcomp,eurosym}
\usepackage{hyperref}
\usepackage{color} 
\def\red#1{\textcolor{red}{#1}}
\name{Chiara Villa} 
%\address{Inria Saclay \\ Postdoc \\ Mathematical Biology} 

\begin{document}

\vspace{-15pt}

%----------------------------------------------------------------------------------------
%	EDUCATION SECTION
%----------------------------------------------------------------------------------------
%\begin{tabular}{ @{} >{\bfseries}l @{\hspace{6ex}} l }
\renewcommand{\arraystretch}{1}
\begin{tabularx}{\linewidth}{>{\hsize=.3\hsize}X> {\hsize=1.7\hsize}X}
%{\bf Email }&  Name [dot] Surname [at] maths [dot] cnrs [dot] fr \\
%{\bf Website} & https://chiaravilla.github.io/website/index.html \\
{\bf Languages} &  Italian, English, French \\%, Brasilian Portuguese (A1) \\
%{\bf Languages} &  Italian (native), English (C2), French (B2)\\%, Brasilian Portuguese (A1) \\
{\bf Software} &  MATLAB, Python, LaTeX, Fortran90, COMSOL,
 Maple, R, HTML5, MS Office \\
{\bf Memberships} & { European Society for Mathematical and Theoretical Biology (ESMTB)}, {Society for Mathematical Biology (SMB),  Société de Mathématiques Appliquées et Industrielles (SMAI), Marie Curie Alumni Association (MCAA)} \\
\end{tabularx}
%\end{tabular}

\begin{rSection}{Academic Appointments}
\underline{\em Current position} \\[3pt]
\noindent
\renewcommand{\arraystretch}{1}
\begin{tabularx}{\linewidth}{>{\hsize=.27\hsize}X> {\hsize=1.73\hsize}X}
{11/25 - \textit{today}} & {\bf Chargée de recherche CNRS}, {\em MAP5, Université Paris Cité}, Paris (FR)\\
\end{tabularx} 

\underline{\em Postdoctoral positions} \\[3pt]
\noindent
\renewcommand{\arraystretch}{1}
\begin{tabularx}{\linewidth}{>{\hsize=.27\hsize}X> {\hsize=1.73\hsize}X}
{02/25 - 08/25} & {\bf Postdoc}, {\em Inria Saclay Centre}, Palaiseau (FR)\\
& {Postdoctoral researcher in the Inria team MUSCA, project OvoTox/OVOPAUSE} \\
%%%%%%%%%
{01/23 - 12/24} & {\bf PRFP postdoc}, {\em Laboratoire Jacques-Louis Lions, Sorbonne Université}, Paris (FR)\\
%& `Partial differential equations for biology: theoretical research, modelling and numerical simulation' (ERC ADORA funding)
%& Postdoctoral researcher in the study of `Partial differential equations for biology: theoretical research, modelling and numerical simulation' (ERC ADORA funding), in the group of Prof Beno{\^{i}}t Perthame
%& {Postdoctoral research fellow of the Paris Region Fellowship Programme (Paris Region and EU Marie Curie funding)} \\
& {Laureate of the Paris Region Fellowship Programme (PR and EU MSCA funding)} \\
%& {Project ``Mechanistic modelling of cell migration and cancer invasion''} \\
{04/22 - 12/22} & {\bf Postdoc}, {\em Laboratoire Jacques-Louis Lions, Sorbonne Université}, Paris (FR)\\
%& `Partial differential equations for biology: theoretical research, modelling and numerical simulation' (ERC ADORA funding)
%& Postdoctoral researcher in the study of `Partial differential equations for biology: theoretical research, modelling and numerical simulation' (ERC ADORA funding), in the group of Prof Beno{\^{i}}t Perthame
& {Postdoctoral researcher in the group of Prof Beno{\^{i}}t Perthame (ERC ADORA funding)} \\
%& {Advisors: Prof Beno{\^{i}}t Perthame, Dr Luís Almeida}  \\
%& {Study: `Partial differential equations for biology: theoretical research, modelling and numerical simulation' (ERC ADORA funding)}  \\
\end{tabularx} 
% \bigskip
\end{rSection}


\begin{rSection}{Education \& Research Experience}

\noindent
\renewcommand{\arraystretch}{1}
\begin{tabularx}{\linewidth}{>{\hsize=.27\hsize}X> {\hsize=1.73\hsize}X}
{09/18 - 03/22} & {\bf PhD, Mathematics}, {\em University of St Andrews}, St Andrews (UK) \\
& {Supervisors: Prof Mark Chaplain, Dr Tommaso Lorenzi}  \\
& {Thesis: `Partial differential equation modelling in cancer and development'} \\
& {\textit{Viva voce} examiners: Prof Philip Maini (UniOx), Dr Nikolaos Sfakianakis (UoStA)} \\
{01/22 - 03/22} & Research visit at Institute Henri Poincaré, Paris (FR)  \\
%& {Funding awarded by the School of Mathematics and Statistics (\pounds 49124.25)} \\
%2018 - 2019 & { Scottish Mathematical Sciences Training Center Graduate courses (Continuum Mechanics, Numerical Methods, Mathematical Biology and Physiology)} \\
%{09/18 - 05/19} & { SMSTC graduate courses (Continuum Mechanics, Numerical Methods, Mathematical Biology and Physiology)} \\

{2014 - 2018} & {\bf MMaths, Applied Mathematics}, {\em University of St Andrews}, St Andrews (UK) \\
& {Fast Track, First Class Honours awarded.  Academic Prizes: Dean’s list (2014-2018), The Principal’s Scholarship for Academic Excellence %(\pounds 1000)
}  \\
%& {Master thesis title: `Mathematical modelling of tumour-induced angiogenesis'} \\
%& {First Class Honours awarded}  \\
%& {Academic Prizes: The Principal’s Scholarship for Academic Excellence, Dean’s list} \\
{Summer 2017} & {{ Undergraduate Summer Research Internship}, {\em University of St Andrews} (UK)}  \\
%{Summer 2017} & {{ \bf Undergraduate Summer Research Internship}, {\em University of St Andrews}}  \\
%& Topic: Mathematical modelling of spatio-temporal evolutionary dynamics of cancer cells focusing on the phenotypic landscape of a solid tumor \\% (Simulations in Matlab) \\
{Summer 2016} & {{Complex Systems Biology Research Internship}, {\em Universit{\`a} degli Studi di Torino} (IT)} \\
%{Summer 2016} & {{\bf Complex Systems Biology Research Internship}, {\em Universit{\`a} degli Studi di Torino}} \\
%& Supervisor: Prof Michele Caselle, Dipartimento di Fisica; Topic: Role of ohnolog genes in regulatory networks, focusing on co-regulation and self-regulation of paralogue pairs \\%(Data analysis in Python) \\
\end{tabularx} 
 
%\bigskip
\end{rSection}


\begin{rSection}{Supervision, Teaching and Marking}

\underline{\em Supervision of postgraduate students} \\[3pt]
\noindent
\renewcommand{\arraystretch}{1}
\begin{tabularx}{\linewidth}{>{\hsize=.27\hsize}X> {\hsize=1.73\hsize}X}
{09/23 - today} & {{\bf PhD thesis supervision} (co-encadrante) of Federica Padovano (LJLL, SU)}\\
{01/23 - 08/23} & {{\bf Master thesis supervision} of Federica Padovano (EPFL), at LJLL (SU)}\\
\end{tabularx} 

\underline{\em Teaching and Marking} \\[3pt]
All activities of 2017-2022 undertaken with the School of Mathematics and Statistics, University of St Andrews. 
%Tutors and demonstrators work through exercise sheets with groups of 11 (tutors) or 50 (demonstrators) students each. 
Teaching activities undertaken with groups of 50 (demonstrating) or 11 (tutoring) students. 
Feedback on Explanation (E), Organisation (O) and Availability (A) on a scale of 1 (excellent) to 5 (poor).
%All available student feedback data from tutorials is included and reported on a scale of 1 (excellent) to 5 (poor) in the categories Explanation (E), Organisation (O) and Availability (A). 

\noindent
\renewcommand{\arraystretch}{1}
\begin{tabularx}{\linewidth}{>{\hsize=.27\hsize}X> {\hsize=1.73\hsize}X}
%{09/18 - 06/22} & {{\bf Mentor in Peer Mentoring scheme} of 4 Undergraduate, 3 Master, 2 PhD students} \\
%{09/18 - today} & \red{{\bf Mentor in Peer Mentoring scheme} of 8 students (4 UG, 3 PGT, 1 PGR)}  \\
{Autumn 2020} & {\bf MT2000 Computing Workshop}, {Demonstrator of computing in Python}\\
{Autumn 2019} & {\bf MT2000 Computing Workshop}, {Demonstrator of computing in Python}\\
{Autumn 2019} & {\bf MT2501 Linear Mathematics}, Tutor of 2 groups 
(E=1.44, O=1.33, A=1.33) \\
{Spring 2019} & {\bf MT2507 Mathematical Modelling}, Tutor of 2 groups (E=1.45, O=1.85, A=1.45)\\
{Spring 2019} & {\bf MT2507 Mathematical Modelling}, Demonstrator of 3 groups\\
{Autumn 2018} & {\bf MT2503 Multivariate Calculus}, Tutor of 2 groups (E=1.17, O=1.5, A=1.17) \\
{Autumn 2018} & {\bf MT2504 Combinatorics and Probability}, Marking of 100 computing projects\\
%{Autumn 2017} & {\bf UK Undergraduate Ambassadors Scheme}, weekly teaching assistance and activities with secondary school pupils (S1, S3, Advanced Higher Maths), {UoSA} module ID4001 - Communication and Teaching in Science, {\em Waid Academy, Anstruther (UK)}
{Autumn 2017} & {\bf UK Undergraduate Ambassadors Scheme}, weekly teaching in secondary school (S1, S3, Advanced Higher Maths), {\em Waid Academy, Anstruther (UK)}
\end{tabularx} 

%\bigskip

\end{rSection}
%-----------------------------%
\newpage

\begin{rSection}{Funding, grants and prizes awarded}

\underline{\em Competitive positions}  \\[3pt]
\noindent
\renewcommand{\arraystretch}{1}
\begin{tabularx}{\linewidth}{>{\hsize=.22\hsize}X> {\hsize=1.78\hsize}X}
{2025 \hfill } & {\bf CNRS Concours Chercheurs 2024}, 
ranked 1st at the concours de Chargé de recherche de classe normale n° 51/02 of the CNRS CDI51 (affiliation Insmi).\\
{2025 \hfill } & {\bf COMPETITION n°9 SIF-CRCN-2025 (INRIA)}, 
ranked 1st at the concours de Chargés de recherche de classe normale n°9-SIF-CRCN-2025 of the Centre INRIA de Saclay -- \textit{renounced to join the CNRS on 01/11/25}.\\
{2025 \hfill } & {\bf MSCA Postdoctoral Fellowships} (EU HORIZON-MSCA-2024-PF-01-01) with the project ``Modelling of Epithelial-to-Mesenchymal Transition in Invading Cells" 
(score: 95.40\%, {\bf \euro 260347.92}) -- \textit{renounced to join the CNRS on 01/11/25}.\\
{2022 \hfill } & {\bf Paris Region Fellowship Programme} (EU Horizon E2020, MSCA Grant Agreement no 945298-ParisRegionFP) with the proposed project on ``Mechanistic models of cell migration and cancer invasion" %: analysis, numerics, validation"
({\bf \euro 257760}).\\
\end{tabularx} 

\underline{\em Additional grants} \\[3pt]
\noindent
\renewcommand{\arraystretch}{1}
\begin{tabularx}{\linewidth}{>{\hsize=.22\hsize}X> {\hsize=1.78\hsize}X}
{2023 \hfill } & {\bf PEPS JCJC} funding awarded by Insmi for the project “Conservative numerical schemes for novel structured PDE models of cancer invasion” with Alexandre Poulain ({\bf \euro 4900}).\\
{2023} & {\bf BOUM SMAI funding}, MC2D workshop organisation, Paris 10/2023 ({\bf \euro 1000}).\\
{2023} & {\bf UFR 929 funding}, Sorbonne Université, MC2D workshop organisation ({\bf \euro 1500}).\\
{2021 \hfill } & {\bf IHP financial support} for the “Mathematical modeling of organization in living matter” thematic program at the Institute Henri Poincaré during 10/01-01/04 2022 ({\bf \euro 4500}).\\
\end{tabularx} 

\underline{\em Smaller awards}  \\[3pt]
\noindent
\renewcommand{\arraystretch}{1}
\begin{tabularx}{\linewidth}{>{\hsize=.22\hsize}X> {\hsize=1.78\hsize}X}
{2022} & {\bf Junior Fellowship} for the participation to the workshop “Parabolic and kinetic models in population dynamics” in Toulouse in September 2022.\\
{2020} & {\bf SMBdevBio Poster Prize 1}, Society for Mathematical Biology, SMB2020 ({\bf \$250}) \\
{2020} & {\bf LMS ECR Travel Grant}, London Mathematical Society, 12th ECMTB ({\bf \pounds 500}) \\
%{2018} & {\bf PhD funding}, School of Mathematics and Statistics, UoStA ({\bf \pounds 49124.25}) \\
{2018} & {\bf The Principal’s Scholarship for Academic Excellence}, prize awarded to the top 50 academically performing students in their final year at the UoStA({\bf \pounds 1000}) \\
{2014 - 2018} & {\bf The Deans' list}, annual award for academic excellence by the Deans of the UoStA  \\
{2017} & {\bf Research scholarship}, Undergraduate Summer Research Internship,  UoStA ({\bf \pounds 1684.29}) 
\end{tabularx} 

\end{rSection}

%-----------------------------%  


\begin{rSection}{Professional responsibilities }
\underline{\em Peer reviewing activity} \\[3pt]
\noindent
\renewcommand{\arraystretch}{1}
\begin{tabularx}{\linewidth}{>{\hsize=.27\hsize}X> {\hsize=1.73\hsize}X}
{03/21 - today} & {\bf Journal Peer Reviewer} (12 manuscripts), {\em PLOS Computational Biology}, {\em Journal of Mathematical Biology},  {\em International Journal of Non-Linear Mechanics},  {\em Bulletin of Mathematical Biology},  {\em iScience}, {\em European Journal of Applied Mathematics}, {\em Mathematical Biosciences},  {\em European Control Conference 2022}, {\em Frontiers in Ecology and Evolution} \\%(Special issue: From Ecology to Cancer Biology and Back Again)\\
\end{tabularx} 

\underline{\em Mentoring and Representation} \\[3pt]
\noindent
\renewcommand{\arraystretch}{1}
\begin{tabularx}{\linewidth}{>{\hsize=.27\hsize}X> {\hsize=1.73\hsize}X}
%{11/20} & {\bf Piscopia Society}, {\em School of Mathematics and Statistics, University of St Andrews}, {\ PhD testimonial to encourage female/non-binary students who are considering a PhD in mathematics, promoting equality and diversity in STEM} \\
{11/25 - today} & {\bf Math C pour L} stage organiser, \textit{MAP5, Universit{\'e} Paris Cit{\'e}}\\
{01/23 - 12/24} & {\bf Member of `Comité Parité'}, \textit{Laboratoire Jacques-Louis Lions, Sorbonne Universit{\'e}}  \\
{10/22 - 12/24} & {\bf Postdoctoral Research Rep}, \textit{Laboratoire Jacques-Louis Lions, Sorbonne Universit{\'e}}  \\
{09/18 - 06/22} & {{\bf Mentor in Peer Mentoring scheme} of 4 Undergraduate, 3 Master, 2 PhD students} \\
{11/20} & {\bf Piscopia Society}, {\em School of Mathematics and Statistics, University of St Andrews}, {\ Testimonial encouraging female/non-binary students considering a PhD in mathematics} \\
%{09/18 - 09/19} & {\bf PGR Rep \& PGR Exec Rep}*   \\
{09/18 - 09/19} & {\bf Postgraduate Research Rep \& Postgraduate Research Executive Rep}, UoStA  \\
%{09/18 - 09/19} & {\bf Postgraduate Research Executive Rep}*\\
{09/18 - 09/19} & {\bf University of St Andrews Student Rep}, {\em SMSTC} \\[5pt]
%\bigskip
\end{tabularx} 

\end{rSection}

\newpage


\begin{rSection}{Selected Scientific Meetings} %Seminars, Conferences, Workshops and Forums
\underline{\em Scientific meetings organised} \\[3pt]
\noindent
\renewcommand{\arraystretch}{1}
\begin{tabularx}{\linewidth}{>{\hsize=.27\hsize}X> {\hsize=1.73\hsize}X}
{06/25} & {\bf Minisymposium} `Mathematical advances in modelling cancer treatment', \textit{SMAI 2025}  \\
{07/24} & {\bf Minisymposium} `Recent advances in modelling cancer invasion', \textit{ECMTB2024}  \\
{10/23 \hfill } & {\bf Workshop `Mathematical challenges in modelling cancer dynamics'} ($\sim$ 50 ppl),  \textit{Laboratoire Jacques-Louis Lions, Sorbonne Universit{\'e}},  https://mc2d.sciencesconf.org/  \\
{09/20 - 12/21} & {\bf StAMBio seminar series},  Weekly talks (online) by members of the St Andrews Mathematical Biology research group and international guest speakers  \\
{01/20} & {\bf Postgraduate Interdisciplinary Mathematics Symposium}, {for PhD students of the School of Mathematics and Statistics of the University of St Andrews}, {\em  Edzell} \\
{12/19} & {\bf Workshop `Scottish Mathematical Biology Forum'}, {\em  ICMS, Edinburgh} \\
\end{tabularx} 

\underline{\em Invitations to speak at International Workshops}  \\[3pt]
\noindent
\renewcommand{\arraystretch}{1}
\begin{tabularx}{\linewidth}{>{\hsize=.21\hsize}X> {\hsize=1.79\hsize}X}
{Dec 2025} & {\bf New trends in mathematical models for biology}, {\em Institut Henri Poincaré} \\
{Apr 2025} & {\bf Young women in Mathematical Biology}, {\em Bonn University} \\
{Apr 2024} & {\bf Mathematical and numerical tools for Oncology}, {\em Oncolille Institut} \\
{Nov 2023} & {\bf Mechanistic models for continuous phenotypic adaptation}, {{\em University of Leeds}} \\
{Jun 2023} & {\bf Mathematical Biology: Analysis and Application}, {{\em Technische Universit{\"a}t Dresden}} \\
{Feb 2023} & {\bf Multiscale analysis and methods for PDEs}, {{\em Institute for Mathematical Sciences} (SG)} \\
{Oct 2022} & {\bf Modelling cell and tissue biomechanics}, {{\em LJLL, Sorbonne University}} \\
{Jun 2021} & {\bf Soft Tissue Mechanics}, {{\em University of St Andrews}, Online} \\
{May 2021} & {\bf Mathematical Biology on the Mediterranean Coast}, {{\em LJLL,  SU}, Online} \\
{Jun 2020} & {\bf Interplay between Oncology, Mathematics and Numerics}, {{\em LJLL, Inserm}, Online} \\
\end{tabularx} 

\underline{\em Invitations to speak at Conference Mini-symposia:}  \\[3pt]
%13th ECMTB (Toledo, 07/24),  ECM (Sevilla, 07/24), SIMAI (Matera, 08/23), 12th ECMTB (Heidelberg, 09/22).
%\begin{tabularx}{\linewidth}{>{\hsize=.21\hsize}X> {\hsize=1.79\hsize}X}
%{Jul 2024} & {\bf 13th ECMTB} (Toledo), \em{Modelling heterogeneity, adaptation and evolution in cancer}\\
%{Jul 2024} & {\bf ECM} (Sevilla), \em{Mathematical methods for biological problems}\\
%{Aug 2023} & {\bf SIMAI} (Matera), \em{Models and methods for biomedical applications}\\
%{Sep 2022} & {\bf 12th ECMTB} (Heidelberg), \em{Evolutionary dynamics in structured populations}\\%: modelling, analytics and numerics}\\
%\end{tabularx} 
\begin{tabularx}{\linewidth}{>{\hsize=.21\hsize}X> {\hsize=1.79\hsize}X}
{Jul 2024} & {\bf 13th ECMTB}, {`Modelling heterogeneity, adaptation and evolution in cancer'}, \textit{Toledo}\\
{Jul 2024} & {\bf ECM}, {`Mathematical methods for biological problems'}, \textit{Sevilla}\\
{Aug 2023} & {\bf SIMAI}, {`Models and methods for biomedical applications'}, \textit{Matera}\\
{Sep 2022} & {\bf 12th ECMTB},  {`Evolutionary dynamics in structured populations'}, \textit{Heidelberg}\\%: modelling, analytics and numerics}\\
\end{tabularx} 

\underline{\em Invitations to speak at Seminars:} \\[3pt]
%Séminaire de modélisation mathématique en sciences de la vie et santé (LJLL, Paris, 11/24), Mathematical Biology Seminar (University of Leeds, 05/24),  Puissant Lab (Saint-Louis Medical Center, Paris, 04/24), The Evolution Seminar (Bielefeld University, 05/23), Analyse Numérique et Équations aux Dérivées Partielles (Université de Lille, 04/23), Synthsys Seminar (Centre for Synthetic and Systems Biology, Edinburgh, 11/22). \\
\begin{tabularx}{\linewidth}{>{\hsize=.21\hsize}X> {\hsize=1.79\hsize}X}
{Nov 2024} & {\bf Séminaire de modélisation mathématique en sciences de la vie et santé}, \textit{LJLL}\\
{May 2024} & {\bf Mathematical Biology Seminar,} \textit{University of Leeds}\\
{Apr 2024} & {\bf Puissant Lab}, \textit{Saint-Louis Medical Center (Paris)}\\
{May 2023} & {\bf The Evolution Seminar,} \textit{Bielefeld University}\\
{Apr 2023} & {\bf Analyse Numérique et Équations aux Dérivées Partielles}, \textit{Université de Lille}\\
{Nov 2022} & {\bf Synthsys Seminar}, \textit{Centre for Synthetic and Systems Biology, Edinburgh}\\
\end{tabularx} 

%\underline{\em Other meetings with talk/poster:} 
%SMAI 2025 (Maubuisson Carcans, 06/25), 
%From Cells to Tissues: Models, Analysis and Applications (Lake Como School of Advanced Studies, 06/24), 
%Neuroscience, Collective Migration and Parameter Estimation (University of Oxford, 07/23), 
%Asymptotic behaviors of systems of PDE arising in physics and biology: theoretical and numerical points of view (Polytech Lille, 06/23), 
%Structured Population Models (University of Warsaw, 05/23), 
%Reaction-Diffusion Network (Université Paris-Saclay, 01/23), 
%Modelling cell and tissue biomechanics (LJLL, 10/22), 
%Mathematical models for bio-medical sciences (LCSoAS, 06/22), 
%Tissue growth and movement (IHP, 01/22), 
%Society for Mathematical Biology (SMB, online, 06/21), 
%Mathematical Population Dynamics, Ecology and Evolution (CIRM, online, 04/21), 
%Postgraduate Interdisciplinary Mathematics Symposium (UoStAnd, online, 01/21), 
%Society for Mathematical Biology (SMB, online, 08/20), 
%School of Mathematics and Statistics Research Day (UoStAnd, 01/20), 
%EMS Postgraduate Meeting (Edinburgh Mathematical Society, 05/19), 
%PDEs in Mathematical Biology: Modelling and Analysis (LMS \& CMI, ICMS, 04/19), 
%British Applied Mathematics Colloquium (University of Bath, 04/19), 
%Postgraduate Interdisciplinary Mathematics Symposium (UoStAnd 01/19), 
%School of Mathematics and Statistics Research Day (UoStAnd, 01/19), 
%Undergraduate Summer Research Conference (UoStAnd, 11/17).\\


%\underline{\em Other meetings attended:} \\
%Morphodynamics of living tissues, April 2024, 
%Institut de France, Paris\\
%Compsyscan22: A complex systems approach to cancer understanding, October 2022, 
%BioSyL, IXXI, PLASCAN, Lyon\\
%Parabolic and kinetic models in population dynamics, September 2022, 
%Institut de Mathématiques de Toulouse, Toulouse (Awarded Junior Fellowship)\\
%BioTOMath Conference: Mathematical Challenges in Biology and Medicine, September 2022, 
%Politecnico di Torino, Torino\\
%Mathematical modeling of organization in living matter, January-March 2022, 
%Institut Henri Poincaré, Paris (Awarded IHP financial support), 
%Thematic trimester program (Mathematical models in ecology and evolution, Mathematical challenges in modelling population dynamics)\\
%Mathematics Challenges in Biology and Medicine, October 2021, 
%Politecnico di Torino, Torino\\
%British Applied Mathematics Colloquium, April 2021, 
%University of Glasgow, Online conference\\
%Modeling, analysis and simulation, November 2019, 
%Laboratoire Jacques-Louis Lions, Sorbonne University, Paris\\
%Computational Approaches in Mathematical Biology, May 2019, 
%University of Dundee, Dundee\\
%Scottish Mathematical Biology Forum, December 2018, 
%Maxwell Institute for Mathematical Sciences, Edinburgh\\
%Scottish Mathematical Training Center Symposium, October 2018, 
%SMSTC, Perth

\end{rSection}


%
%%\begin{rSection}{Publications}
%\begin{rSection}{Selected research outputs}
%{\bf Preprints}\\[2pt]
%{\bf [14]} B. Perthame, C. Villa,  Regularity and stability in a strongly degenerate nonlinear diffusion and haptotaxis model of cancer invasion, 2024.  \textbf{arXiv:2412.18261}, \textbf{hal-04854773}\\
%{\bf [13]} T. Lorenzi, N. Loy, C. Villa,  Phenotype-structuring of non-local kinetic models of cell migration driven by environmental sensing, 2024.  \textbf{arXiv:2412.16258}, \textbf{hal-04851469}\\
%{\bf [12]} S. Hamis, A.P. Browning, A.L. Jenner, C. Villa, P. K. Maini and T. Cassidy, Growth rate-driven modelling reveals how phenotypic adaptation drives drug resistance in BRAFV600E-mutant melanoma 2024. \textbf{bioRxiv 2024.08.14.607616}, \textbf{hal-04851795} \\
%{\bf [11]} L. Almeida, A. Poulain, A. Pourtier, C. Villa,  Mathematical modelling of the contribution of senescent fibroblasts to basement membrane digestion during carcinoma invasion, 2024. \textbf{hal-04574340} \\[6pt]
%{\bf Papers published in peer-reviewed international journals}\\[2pt]
%{\bf [10]} T. Lorenzi, K.J. Painter, C. Villa,  Phenotype structuring in collective cell migration: a tutorial into mathematical models and methods, \textit{Journal of Mathematical Biology}, 91(2):22, 2025.  DOI: 10.1007/s00285-025-02246-5, \textbf{hal-04851615}\\
%{\bf [9]} A.P. Browning, R. Crossley, C. Villa, P. K. Maini, A.L. Jenner, T. Cassidy and S. Hamis, Identifiability of heterogeneous phenotype adaptation from low-cell-count experiments and a stochastic model, \textit{PLOS Computational Biology,} 21(6):e1013202, 2025.  DOI: 10.1371/journal.pcbi.1013202, \textbf{hal-04854906} \\
%{\bf [8]} C. Villa, P. K. Maini, A.P. Browning, A.L. Jenner, S. Hamis and T. Cassidy, Reducing phenotype-structured PDE models of cancer evolution to systems of ODEs: a generalised moment dynamics approach, \textit{Journal of Mathematical Biology}, 91(2):22, 2025. DOI: 10.1007/s00285-025-02246-5, \textbf{hal-04599519} \\
%{\bf [7]} F.  Padovano, C. Villa, The development of drug resistance in metastatic tumours under chemotherapy: an evolutionary perspective, {\em Journal of Theoretical Biology}, 595(1):111957, 2024. DOI: 10.1016/j.jtbi.2024. 111957, \textbf{hal-04595087v3} \\
%{\bf [6]} L. Almeida, J.A. Denis, N. Ferrand, T. Lorenzi, M. Sabbah, C. Villa, Evolutionary dynamics of glucose-deprived cancer cells: insights from experimentally-informed mathematical modelling, \textit{Journal of the Royal Society Interface}, 21(210):20230587,  2024. DOI: 10.1098/rsif.2023.0587, \textbf{hal-03947209v2}. \\
%{\bf [5]} {C. Villa}, A. Gerisch, M.A.J. Chaplain, A novel nonlocal partial differential equation model of endothelial progenitor cell cluster formation during the early stages of vasculogenesis, {\em Journal of Theoretical Biology}, 534(1):110963, 2022. DOI:
%10.1016/j.jtbi.2021.110963, \textbf{hal-04415625}. 
%\\
%{\bf [4]} F. Mottes, C. Villa, M. Osella, M. Caselle, The impact of whole genome duplications on the human gene regulatory networks, {\em PLOS Computational Biology}, 17(12):e1009638, 2021. DOI: 10.1371/journal.pcbi.1009638  \textbf{hal-04415666}.
% \\
%%{\bf Papers submitted (under peer revision)}\\
%{\bf [3]}  {C. Villa}, M.A.J. Chaplain, A. Gerisch, T. Lorenzi, {Mechanical models of pattern and form in biological tissues: the role of stress-strain constitutive equations}, {\em Bulletin of Mathematical Biology}, 83:80, 2021. DOI: 10.1007/s11538-
%021-00912-5, \textbf{hal-04415645}. 
%\\
%{\bf [2]} {C. Villa}, M.A.J. Chaplain, T. Lorenzi, {Evolutionary dynamics in vascularised tumours under chemotherapy: Mathematical modelling, asymptotic analysis and numerical simulations}, {\em Vietnam Journal of Mathematics}, 49, 143–167, 2021. DOI: 10.1007/s10013-020-00445-9, \textbf{hal-04415601}. 
%\\
%{\bf [1]} {C. Villa}, M.A.J. Chaplain, T. Lorenzi, {Modelling phenotypic heterogeneity in vascularised tumours}, {\em SIAM Journal on Applied Mathematics}, 81, 434–453, 2021. DOI: 10.1137/19M1293971,  \textbf{hal-04415631}. 
%%\\%[4pt]
%%%{\bf Conference proceedings}\\[2pt]
%%{\bf [1]} T. Lorenzi, F.R. Macfarlane, {C. Villa}, {Discrete and continuum models for the evolutionary and spatial dynamics of cancer: a very short introduction through two case studies}, (pp. 359-380) in {\em Trends in Biomathematics: Modeling Cells, Flows, Epidemics, and the Environment}, Ed. R. Mondaini, Springer, Cham, 2019. DOI: 10.1007/978-3-030-46306-9\_22, \textbf{hal-04415585}. 
%%\\[4pt]
%%{\bf Doctoral thesis}\\[2pt]
%%{\bf [T1]} {C. Villa}, Partial differential equation modelling in cancer and development, PhD thesis, University of St Andrews, St Andrews, 2022. HAL Id: \textbf{tel-04442733}. 
%%%\\[4pt]
%\end{rSection}


\vfill

{\em Latest update on \today} 


\end{document}
